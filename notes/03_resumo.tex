\section{Resumo}

A otimização de geometria molecular é um tópico essencial da química
computacional, em que utilizando funções de Superfície de Energia Potencial (SEP) é possível 
predizer configurações estáveis de uma molécula ou estados de transição. 
O objetivo deste projeto foi desenvolver um algoritmo para otimização de funções SEP baseado no método
de Newton, mantendo uma taxa de convergência similar e com menor custo computacional. 
O algoritmo proposto, CBPD (Convergence Based on Partial Derivatives), foi aplicado à reação 
\ce{F + H_2O -> FH + HO}, permitindo avaliar sua eficácia em diferentes cenários de otimização. 
Comparado ao método de Newton, o CBPD mostrou boas taxas de convergência em cenários específicos,
apesar de exigir um número maior de iterações cada etapa de iteração possui um menor custo computacional.
Os resultados indicam que o CBPD pode ser uma alternativa promissora para problemas de otimização em 
química computacional de funções SEP em um espaço de alta dimensionalidade.
