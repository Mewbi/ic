\section{Introdução}

A otimização geométrica de estruturas moleculares é um ramo de pesquisa da química
computacional que busca por meio de algoritmos identificar conformações moleculares que
são estáveis, ou que sejam uma configuração de uma etapa de uma reação química. Tais
conformações são relevantes de serem conhecidas não somente para ter conhecimento sobre
como funcionam as reações, mas também podem ser aplicadas em um novo contexto, como
por exemplo, a criação de novos fármacos.

Toda molécula possui uma Superfície de Energia Potencial (PES, do inglês Potential
Energy Surface), variando a configuração da molécula (mudando o tamanho das ligações ou
os ângulos entre ligações) irá proporcionar variações na energia necessária para manter a
molécula naquele estado. Dessa forma, a otimização geométrica busca identificar os pontos
estacionários na PES, que representam as conformações estáveis da molécula ou estados de
transição. [1]

Existem diversos métodos utilizados na química computacional para otimizar uma
função, sendo um dos conceitualmente mais simples o Método de Newton que é um processo
iterativo que converge para máximos locais, mínimos locais ou pontos de sela, que são
quando o gradiente da função é zero, ou seja, quando todas as suas derivadas parciais são
zero [4] . Outro método que vale destaque é o Método da Secante para localizar raízes de uma
função unidimensional. Ele pode ser utilizado para evitar o uso da derivada da função no
Método de Newton de identificação de raízes, o que é conveniente para evitar o custo
computacional de se calcular a derivada de uma função, que costuma ser alto [5] .

Um algoritmo que identifica as raízes das derivadas parciais de uma função é um
algoritmo de otimização, que pode ser utilizado para otimizar funções de PES e obter a
conformação molecular nesses estados [1] . O objeto de estudo do projeto será a construção de
um algoritmo de otimização de funções PES. A estratégia será utilizar o Método de Newton
de localização de raízes de uma função unidimensional adaptando para funções
multidimensionais, utilizando o Método da Secante para evitar o custo computacional do
cálculo da derivada analítica utilizado em cada etapa do Método de Newton.

O algoritmo será utilizado para convergir para um ponto estacionário da função PES
que descreve a reação entre flúor e água (\ce{F + H2O -> FH + HO}) [2,3] e assim mediar a sua performance.
Espera-se que capacidade de convergência, isto é, a velocidade, precisão e número de
iterações necessárias para convergir uma função deva ser similar ao Método de Newton que
utiliza-se dos gradiente e hessianas da função para realizar a otimização da função.
