\section{Introdução}

A otimização geométrica de estruturas moleculares é um ramo de pesquisa da química
computacional que busca por meio de algoritmos identificar conformações moleculares que
são estáveis, ou que sejam uma configuração de uma etapa de uma reação química. É de interesse compreender essas conformações não apenas para o entendimento dos mecanismos das reações, mas também para aplicações práticas, como o desenvolvimento de novos fármacos.

O conceito de Superfície de Energia Potencial (do inglês, \textit{Potential Energy Surface - PES}) descreve que para cada configuração de uma molécula, i.e, o tamanho das ligações e ângulos em que os átomos estão arranjados, existe um valor de de energia associado. Dado que uma PES pode estar associada a uma função, localizar os mínimos locais dessa função (pontos estacionários), representa localizar configuções de moléculas com menores valores de energia associado. Configurações estáveis de uma molécula ou estados de transição costumam estar associados com menos valores de energia. % Acho que da para referenciar algo aqui


% Falar de métodos iterativos
% Abordar que funções PES não costumam ter expressões analiticas e que os valores das derivadas tendem a ser calculados por aproximação.



%%%%% Trocar
Existem diversos métodos utilizados na química computacional para otimizar uma
função, sendo um dos conceitualmente mais simples o Método de Newton que é um processo
iterativo que converge para máximos locais, mínimos locais ou pontos de sela, que são
quando o gradiente da função é zero, ou seja, quando todas as suas derivadas parciais são
zero [4] . Outro método que vale destaque é o Método da Secante para localizar raízes de uma
função unidimensional. Ele pode ser utilizado para evitar o uso da derivada da função no
Método de Newton de identificação de raízes, o que é conveniente para evitar o custo
computacional de se calcular a derivada de uma função, que costuma ser alto [5] .
%%%%%


O objetivo de estudo do projeto é a construção de um algoritmo baseado no Método de Newton e Método da Secante, que permita realizar a otimização de funções PES. O algoritmo trata o processo de otimização de uma função multidimensional como uma soma de funções unidimensionais. No caso, é feito a otimização de cada derivada parcial da função multidimensional utilizando o Método de Newton e aproximando os valores das derivadas parciais utilizando o Método da Secante.

% Falar com qual reação que o projeto está trabalhando.

%%%%% Trocar
Um algoritmo que identifica as raízes das derivadas parciais de uma função é um
algoritmo de otimização, que pode ser utilizado para otimizar funções de PES e obter a
conformação molecular nesses estados [1] . O objeto de estudo do projeto será a construção de
um algoritmo de otimização de funções PES. A estratégia será utilizar o Método de Newton
de localização de raízes de uma função unidimensional adaptando para funções
multidimensionais, utilizando o Método da Secante para evitar o custo computacional do
cálculo da derivada analítica utilizado em cada etapa do Método de Newton.

O algoritmo será utilizado para convergir para um ponto estacionário da função PES
que descreve a reação entre flúor e água (\ce{F + H2O -> FH + HO}) [2,3] e assim mediar a sua performance.
Espera-se que capacidade de convergência, isto é, a velocidade, precisão e número de
iterações necessárias para convergir uma função deva ser similar ao Método de Newton que
utiliza-se dos gradiente e hessianas da função para realizar a otimização da função.
%%%%

% Falo como o resumo é organizado aqui?
% Vi que isso eh usado em artigos, mas não sei se faz sentido no resumo, já que ele ja possui sumário.
