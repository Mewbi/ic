\section{Introdução}

A otimização geométrica de estruturas moleculares é um ramo de pesquisa da química
computacional que busca por meio de algoritmos identificar conformações moleculares que
são estáveis, ou que sejam uma configuração de uma etapa de uma reação química. É de interesse compreender essas conformações não apenas para o entendimento dos mecanismos das reações, mas também para aplicações práticas, como o desenvolvimento de novos fármacos.

O conceito de Superfície de Energia Potencial (SEP) (do inglês, \textit{Potential Energy Surface}) descreve que para cada configuração de uma molécula, i.e, o tamanho das ligações e ângulos em que os átomos estão arranjados, existe um valor de energia associado. Dado que uma SEP pode estar associada a uma função, localizar os mínimos locais dessa função (pontos estacionários), representa localizar configurações de moléculas com menores valores de energia associado. Configurações estáveis de uma molécula ou estados de transição costumam estar associados com menores valor de energia. % TODO: buscar artigo que fale sobre SEP, não usar a do fh2o

Funções que representam SEP, por conta de suas complexidades, não costumam possuir expressões analíticas. Dessa maneira, os valores de suas derivadas, que são frequentemente utilizados em métodos de otimização, passam a ser obtidos  exclusivamente por aproximações. Diante disso, é relevante que o método de otimização implementado possua uma maneira eficiente de fazer o cálculo das derivadas.

O objetivo de estudo do projeto é a construção de um algoritmo baseado no Método de Newton e Método da Secante, que permita realizar a otimização de funções SEP. O algoritmo trata o processo de otimização de uma função multidimensional como uma soma de funções unidimensionais. No caso, é feito a otimização de cada derivada parcial da função multidimensional utilizando o Método de Newton e aproximando os valores das derivadas parciais utilizando o Método da Secante.

Para a validação do algoritmo desenvolvido será utilizado a função SEP que descreve a reação entre fluor e água (\ce{F + H2O -> FH + HO}) \cite{fh2o_first_sep}. O algoritmo desenvolvido denominado CBPD (\textit{Convergence Based in Partial Derivatives}) será comparado com o Método de Newton em diferentes cenários de convergência, avaliando a taxa de convergência e a quantidade de iterações necessárias para convergir.

A organização do restante deste resumo é descrita a seguir. A Seção \ref{sec:definitions} apresenta termos fundamentais em métodos de otimização. A Seção \ref{sec:teorical-base} revisa os métodos de convergência que serão utilizados como base, assim como descreve sobre a reação de estudo. A metodologia utilizada nesse projeto é descrita na Seção \ref{sec:methodology}. Por fim os resultados e conclusões são apresentados nas Seções \ref{sec:results} e \ref{sec:conclusions} respectivamente.
