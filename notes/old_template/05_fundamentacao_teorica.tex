\section{Fundamentação Teórica}
\label{sec:teorical-base}
  O processo de otimização pode ser realizado com diferentes métodos, que nesse contexto possuem o mesmo objetivo, localizar as raízes de uma função, seja essa função fornecida analiticamente ou numericamente. Todo método existente possui seus pontos positivos e negativos dentre os demais, seja por sua eficiência, facilidade de implementação ou custo computacional relacionado a cada etapa da iteração. A pesquisa está se embasando principalmente sobre o Método de Newton, que satisfazendo seus critérios de convergência$^{ \cite{calculo_numerico_aplicado} }$ é classificado como estável, ou seja, a cada iteração é obtido um novo ponto mais próximo da região de convergência que possui um erro relativo menor quando comparado com a etapa anterior.

\subsection{Método de Newton}
\label{sec:newton-method}

O Método de Newton, dado um ponto inicial, utilizando a derivada da função em estudo, se obtém um próximo ponto mais próximo da região de convergência. Nesse processo, iniciamos de um ponto inicial arbitrário da função e traçamos uma reta tangente da função no ponto, ou seja, verificamos o valor da derivada no ponto e utilizamos para definir uma função afim $g: {\mathds{R}\to\mathds{R}}$ com $g(x) = ax + b$, $a, b \in \mathds{R}$. Com a função afim definida no ponto, identificamos a raiz da função, ou seja, definimos o ponto em que $g(x)=0$.

Podemos calcular o ponto $x_n$ em que $g(x) = 0$ da seguinte maneira:
%
\begin{equation}
  x_n = x_{n-1} - \frac{f(x)}{f'(x)}
\end{equation}
%
Realizamos novamente o processo anterior, agora para o ponto $f(x_n)$. Para cada novo ponto $x_k$ obtido aproximamos cada vez mais do ponto em que $f(x_k)$ é igual a zero.


\subsection{Método da Secante}

O Método da Secante, utiliza dois pontos da função para calcular o próximo ponto da iteração com o objetivo que seja mais próximo da raíz da função. Cada iteração é calculada com base na reta formada pelos dois pontos anteriores, que será secante ao gráfico da função. Dessa forma, o ponto $x_n$ é definido:
%
\begin{equation}
  x_n = x_{n-1} - f(x_{n-1}) \frac{x_{n-1} - x_{n-2}}{f(x_{n-1}) - f(x_{n-2})}
\end{equation}
%
Trazendo enfoque na etapa de iteração, mais precisamente no termo relativo a reta secante
%
\begin{equation}
  \frac{x_{n-1} - x_{n-2}}{f(x_{n-1}) - f(x_{n-2})} \,.
\end{equation}
%
se definirmos $x_{n-1} = x_{n-2} + h$ com $h \in \mathds{R}$ temos
%
\begin{equation}
  \frac{x_{n-2} + h - x_{n-2}}{f(x_{n-2} + h) - f(x_{n-2})} = \frac{h}{f(x_{n-2} + h) - f(x_{n-2})} \,.
\end{equation}
%
Podemos tomar o inverso dessa expressão, e, para casos em quem que a distância dentre os dois pontos seja suficientemente pequena, ou seja, caso o valor de $h$ tenda a $0$, o termo representado passa a ser o inverso de uma aproximação da derivada no ponto.
%
\begin{equation}
  \Bigg( \frac{f(x_{n-2} + h) - f(x_{n-2})}{h} \Bigg)^{-1} \approx \Bigg( \lim\limits_{h\to0}\frac{f(x_{n-2}+h)-f(x_{n-2})}{h} \Bigg)^{-1} = \Bigg( \frac{df}{dx}(x_{n-2}) \Bigg)^{-1}
\end{equation}
%
Considerando que a função $f$ é diferenciável, o Teorema do Valor Médio$^{ \cite{calculo_1} }$ afirma que existe uma reta tangente entre os dois pontos $x_{n-1}$ e $x_{n-2}$ que o seu valor é exatamente o valor da reta secante calculada.

\subsection{Método de Newton Multidimensional}

O Método de Newton Multidimensional consiste em uma generalização do Método de Newton porém para casos que possam envolver funções multidimensionais. Supondo uma função $F: {\mathds{R}^k\to\mathds{R}^k}$ com $k \in \mathds{N}$ sendo $\mathbf{x}_n \in \mathds{R}^k$ referente ao enésimo ponto do processo de otimização. Um novo ponto $\mathbf{x}_{n+1}$ é definido:
%
\begin{equation}
  \label{eq:newton_generalization}
  \mathbf{x}_{n+1} = \mathbf{x}_n - J_F(\mathbf{x}_n)^{-1}F(\mathbf{x}_n) \,,
\end{equation}
%
sendo $J_F(\mathbf{x}_n)^{-1}$ a matriz inversa $k \times k$ do Jacobiano da função $F$.

É importante se atentar nesse método sobre quais condições a etapa de iteração pode performar. Nesse caso é sempre necessário verificar se a matriz $J_F(\mathbf{x}_n)$ é inversível, ou seja, $\det(J_F(\mathbf{x}_n)) \neq 0$, pois nesses casos não é possível dar sequência no processo de otimização.

\subsection{Reação \ce{F + H2O -> FH + HO}}

Nessa pesquisa será estudada a reação de \ce{F + H2O -> FH + HO} que passa por 5 pontos estacionários. Cada ponto estacionário possui uma conformação específica que permitem que os reagentes interajam. No caso em estudo são:
%
\begin{itemize}[itemsep=0pt,parsep=0pt]
  \item Reagentes
  \item R-vdW
  \item TS
  \item P-vdW
  \item Produtos
\end{itemize}

% \begin{wrapfigure}{r}{0.6\textwidth}
\begin{figure}
  \begin{center}
    \pgfdeclarelayer{background}
\pgfsetlayers{background,main}

\newcommand{\Bond}[6]%
% start, end, thickness, incolor, outcolor, iterations
{ 
  \begin{pgfonlayer}{background}
    \colorlet{InColor}{#4}
    \colorlet{OutColor}{#5}
    \foreach \I in {#6,...,1}
      {
        \pgfmathsetlengthmacro{\r}{#3/#6*\I}
        \pgfmathsetmacro{\C}{sqrt(1-\r*\r/#3/#3)*100}
        \draw[InColor!\C!OutColor, line width=\r] (#1.center) -- (#2.center);
      }
  \end{pgfonlayer}
}

\newcommand{\SingleBond}[2]%
% start, end
{   
  \Bond{#1}{#2}{0.7mm}{white}{black!75}{10}
}

\begin{center}
  
  \begin{tikzpicture}[
      scale = 1,
      plateau/.style = {
        draw,
        thick,
        rectangle,
        minimum width = 1cm,
        inner sep = 0pt
      },
      oxygen/.style = {
        circle,
        ball color=red,
        minimum size=4mm,
        inner sep=0
      },
      hydrogen/.style = {
        circle, 
        ball color=white, 
        minimum size=3mm, 
        inner sep=0
      },
      fluorine/.style = {
        circle, 
        ball color=cyan!30,
        minimum size=3.5mm, 
        inner sep=0
      }
    ]
    % \draw[step=1cm,gray,very thin] (0,0) grid (12,8);

    % Energy pleateau 
    \node[plateau] (reactant) at (1.5, 5) {};
    \node[plateau] (rvdw) at (4, 4) {};
    \node[plateau] (ts) at (6.5, 6) {};
    \node[plateau] (pvdw) at (9, 2) {};
    \node[plateau] (product) at (11.5, 3) {};

    % Lines between pleateau
    \draw[dashed, thick] (reactant.east) -- (rvdw.west);
    \draw[dashed, thick] (rvdw.east) -- (ts.west);
    \draw[dashed, thick] (ts.east) -- (pvdw.west);
    \draw[dashed, thick] (pvdw.east) -- (product.west);

    % Pleateau legends
    \node[] (reactantName) at ($(reactant) + (0, -1.1)$) {\small Reactant};
    \node[] (reactantValue) at ($(reactant) + (-1, 0)$) {0.0};
    
    \node[] (rvdwName) at ($(rvdw) + (0, -0.8)$) {\small R-vdW};
    \node[] (rvdwValue) at ($(rvdw) + (0, -0.3)$) {-3.371};
    
    \node[] (tsName) at ($(ts) + (0, -0.8)$) {\small TS};
    \node[] (tsValue) at ($(ts) + (0, -0.3)$) {1.499};
    
    \node[] (pvdwName) at ($(pvdw) + (0, -0.5)$) {\small P-vdW};
    \node[] (pvdwValue) at ($(pvdw) + (-1.4, 0.0)$) {-22.337};
    
    \node[] (productName) at ($(product) + (0, -0.9)$) {\small Product};
    \node[] (productValue) at ($(product) + (1.3, 0.0)$) {-16.151};
    
    % Molecules
    %% Reactant
    \node[oxygen] (1O1) at ($(reactant) + (0, -0.7)$) {};
    \node[hydrogen] (1H1) at ($(reactant) + (-0.4, -0.4)$) {};
    \node[hydrogen] (1H2) at ($(reactant) + (0.4, -0.4)$) {};
    \node[fluorine] (1F1) at ($(reactant) + (0, 0.4)$) {};
    
    \SingleBond{1O1}{1H1}
    \SingleBond{1O1}{1H2}

    %% R-vdW
    \node[hydrogen] (2H1) at ($(rvdw) + (0.1, 1.2)$) {};
    \node[hydrogen] (2H2) at ($(rvdw) + (0.3, 0.4)$) {};
    \node[oxygen] (2O1) at ($(rvdw) + (0.4, 0.9)$) {};
    \node[fluorine] (2F1) at ($(rvdw) + (-0.4, 0.4)$) {};
    
    \SingleBond{2O1}{2H1}
    \SingleBond{2O1}{2H2}
    
    %% TS
    \node[oxygen] (3O1) at ($(ts) + (0.4, 0.35)$) {};
    \node[hydrogen] (3H1) at ($(ts) + (0.5, 0.8)$) {};
    \node[hydrogen] (3H2) at ($(ts) + (-0.1, 0.4)$) {};
    \node[fluorine] (3F1) at ($(ts) + (-0.6, 0.5)$) {};
    
    \SingleBond{3O1}{3H1}
    
    %% P-vdW
    \node[oxygen] (4O1) at ($(pvdw) + (0.5, 1)$) {};
    \node[hydrogen] (4H1) at ($(pvdw) + (0.9, 0.8)$) {};
    \node[hydrogen] (4H2) at ($(pvdw) + (-0.1, 0.6)$) {};
    \node[fluorine] (4F1) at ($(pvdw) + (-0.35, 0.3)$) {};
    
    \SingleBond{4O1}{4H1}
    \SingleBond{4F1}{4H2}

    %% Product
    \node[oxygen] (5O1) at ($(product) + (-0.2, -0.4)$) {};
    \node[hydrogen] (5H1) at ($(product) + (0.2, -0.4)$) {};
    \node[hydrogen] (5H2) at ($(product) + (0.2, 0.4)$) {};
    \node[fluorine] (5F1) at ($(product) + (-0.2, 0.4)$) {};
    
    \SingleBond{5O1}{5H1}
    \SingleBond{5F1}{5H2}

  \end{tikzpicture}
\end{center}

  \end{center}
  \caption{Representação gráfica do perfil SEP da reação \ce{F + H2O -> FH + HO}.}
  \label{fig:perfil_sep_fh2o}
\end{figure}
% \end{wrapfigure}

Cada ponto estacionário é caracterizado por uma geometria específica que possui um valor de energia associado, que é uma consequência da conformação geométrica dos átomos e suas interações. Esses átomos ficam configurados de maneira que proporcionam a energia mínima para que cada ponto estacionário da reação ocorra. Possuindo a função que descreve a energia de cada conformação de uma dada reação, é possível determinar a configuração geométrica ótima para cada ponto estacionário localizando o mínimo local da função.

\subsection{Módulo Fortran para Função SEP}
\label{sec:module_pes}

Nessa pesquisa, será utilizado um módulo implementado em Fortran$^{ \cite{fh2o_sep_fortran_module} }$, o qual possui uma interface em Python. Essa interface permite a inserção de configurações geométricas da reação \ce{F + H2O -> FH + HO} como entrada e, como resultado, retorna o valor da energia associada a essa configuração. A função em Python recebe como parâmetro uma lista de tamanho 6, $\mathbf{x}_n \in \mathds{R}^6$, que representam cada configuração da reação e recebe como retorno o valor de energia associado em kcal/mol.

\begin{table}[h]
    \centering
    \caption{Relação entre variáveis esperadas pela função SEP e quais coordenadas representam na reação \ce{F + H2O -> FH + HO}.}
    \label{tab:configs}
    \begin{tabular}{@{}cc@{}}
    \hline
    Variável & Coordenada \\
    \hline
      $x_1$ & Distância \ce{H-O} \\
      $x_2$ & Distância \ce{O-H$'$} \\
      $x_3$ & Distância \ce{H$'$-F} \\
      $x_4$ & Ângulo \ce{HOH$'$} \\
      $x_5$ & Ângulo \ce{OH$'$F} \\
      $x_6$ & Ângulo Diedro \ce{HOH$'$F} \\
    \hline
    \end{tabular}
\end{table}
