\section{Conclusões}
\label{sec:conclusions}

A presente pesquisa permitiu explorar um novo método iterativo aplicado em um cenário real, de maneira comparativa com o método de Newton e permitindo evidenciar resultados positivos e negativos em diferentes cenários de convergência, avaliando seu desempenho baseando-se principalmente na quantidade de iterações necessárias para convergir.

Apesar do Método CBPD possuir uma quantidade de iterações necessárias para convergir próximas do dobro de iterações quanto comparado com o Método de Newton, é válido levar em consideração o custo computacional envolvido no processamento de ambos os algoritmos. No método CBPD, para cada iteração possui uma complexidade $O(n)$ para ser calculada, considerando $n$ a quantidade de parâmetros da função a ser otimizada. Já no Método de Newton, possui uma complexidade $O(n^3)$ decorrente da necessidade de inversão da matriz hessiana $n \times n$.

É válido seguir os estudos nessa pesquisa para analisar os critérios formais de convergência para o método caso existam, que podem estar relacionados com casos em que a matriz hessiana é diagonalizável e o valor dos elementos de sua matriz diagonal seja suficientemente próximos dos autovetores da matriz. Esse entendimento permite predizer com mais assertividade os cenários em que é vantajoso a utilização do método.

Ademais é relevante analisar especificamente o caso de convergência para o ponto estacionário P-vdW, em que o método CBPD apresentou uma taxa de convergência consideravelmente superior ao do método de Newton, resultado esse não esperado, visto que o método de Newton tende a ser mais estável nos casos de convergência.
